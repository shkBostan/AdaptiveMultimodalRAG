\documentclass[11pt,a4paper]{article}

% Packages
\usepackage[utf8]{inputenc}
\usepackage[T1]{fontenc}
\usepackage{graphicx}
\usepackage{amsmath}
\usepackage{amssymb}
\usepackage{amsthm}
\usepackage{algorithm}
\usepackage{algorithmic}
\usepackage{booktabs}
\usepackage{multirow}
\usepackage{url}
\usepackage{hyperref}
\usepackage{xcolor}
\usepackage{subcaption}
\usepackage{natbib}
\usepackage{geometry}

% Page geometry
\geometry{margin=1in}

% Hyperref settings
\hypersetup{
    colorlinks=true,
    linkcolor=blue,
    citecolor=blue,
    urlcolor=blue,
    pdftitle={AdaptiveMultimodalRAG: A Framework for Multimodal Retrieval-Augmented Generation},
    pdfauthor={s Bostan}
}

% Theorem environments
\theoremstyle{definition}
\newtheorem{definition}{Definition}
\theoremstyle{plain}
\newtheorem{theorem}{Theorem}
\newtheorem{lemma}{Lemma}

% Title information
\title{AdaptiveMultimodalRAG: A Framework for Multimodal Retrieval-Augmented Generation}
\author{s Bostan}
\date{\today}

\begin{document}

\maketitle

\begin{abstract}
% TODO: Write abstract (150-250 words)
% Key points to cover:
% - Problem: limitations of text-only RAG for multimodal content
% - Approach: adaptive multimodal fusion strategies
% - Contributions: framework, experiments, results
% - Impact: improved retrieval and generation quality
\end{abstract}

% Keywords (uncomment and modify based on target journal requirements)
% \begin{keywords}
% Retrieval-Augmented Generation, Multimodal Learning, Information Retrieval, 
% Embeddings, Document Understanding
% \end{keywords}

% Main content
\input{sections/introduction}
\input{sections/related_work}
\input{sections/methodology}
\input{sections/dataset}
\input{sections/experiments}
\input{sections/discussion}
\input{sections/conclusion}

% References
\bibliographystyle{plainnat}
\bibliography{references}

\end{document}

